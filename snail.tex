\startluacode
local 配置 = {}
配置.字号 = "BodyFontSize" -- ConTeXt 正文字体所用字号
配置 = {
    字号 = 配置.字号,
    文字 = {颜色 = "black"},
    框形 = "fullsquare",
    框 = {余地 = 配置.字号, 线宽 = ".175" .. 配置.字号, 颜色 = "darkgray", 各向同性 = "false"},
    背景 = {颜色 = "lightgray"},
    路径 = {线宽 = ".2" .. 配置.字号, 颜色 = "darkgray", 圆角 = 配置.字号},
    玩笑 = "0"
}
table.save("snail.conf", 配置)
\stopluacode

\startMPinclusions
lua("配置 = table.load('snail.conf')");
def 获 expr a = lua("mp.print(" & a & ")") enddef;
def 设 expr a = lua(a) enddef;

def 恢复默认配置 =
  lua("配置 = table.load('snail.conf')")
enddef;
\stopMPinclusions

\startMPinclusions[+]
def 宫 (suffix name) (expr a) =
  picture name;
  begingroup
    save 框, 文, w, h, s;
    path 框; picture 文; numeric w, h, s;
    文 := textext(a);
    w := bbwidth 文; h := bbheight 文;
    if (获 "配置.框.各向同性"):
      if w > h: s = w; else: s = h; fi;
      框 := fullsquare xysized (s, s) enlarged ((获 "配置.框.余地") * (1, 1));
    else:
      框 := fullsquare xysized (w, h) enlarged ((获 "配置.框.余地") * (1, 1));
    fi;
    框 := (获 "配置.框形") xysized (bbwidth 框, bbheight 框);
    if (获 "配置.玩笑") > 0: 框 := 框 randomized (获 "配置.玩笑"); fi;
    name := image (fill 框 withcolor (获 "配置.背景.颜色");
                   draw 框 withpen pencircle scaled (获 "配置.框.线宽")
                           withcolor (获 "配置.框.颜色");
                   draw 文 withcolor (获 "配置.文字.颜色"););
  endgroup;
enddef;
\stopMPinclusions

\defineframed[snailframed][offset=1em,frame=off]
\startMPinclusions[+]
  def 廷 (suffix name) (expr a) =
    picture name;
    begingroup
    save 文; picture 文; 文 := textext("\snailframed{" & a & "}");
    name := image (draw 文 withcolor (获 "配置.文字.颜色"););
    endgroup;
enddef;
\stopMPinclusions

\startMPinclusions[+]
def 宫廷 (suffix a) (expr b) = picture a; a := b; enddef;
def 城 (suffix a) (text b) =
  宫廷(a, nullpicture);
  for i = b: addto a also image(draw i); endfor;
enddef;
\stopMPinclusions

\startMPinclusions[+]
def 景物 (suffix name) (expr a, w, h) =
  picture name;
  begingroup;
  save p, s; picture p; numeric s;
  p := externalfigure a;
  s := (bbwidth p) / (bbheight p);
  if numeric w and numeric h:
    name := image(draw externalfigure a xysized (w, h););
  else:
    if numeric w:
      name := image(draw externalfigure a xysized (w, w / s););
    fi;
    if numeric h:
       name := image(draw externalfigure a xysized (h * s, h););
    fi;
  fi;
  endgroup;
enddef;
\stopMPinclusions
  
\startMPinclusions[+]
def 显 = true enddef;
def 隐 = false enddef;
def 四角十二门 (suffix foo) (expr 或显或隐) = 
  forsuffixes i = 东北角, 东南角, 西南角, 西北角,
                  子门, 卯门, 午门, 酉门,
                  丑门, 寅门, 辰门, 巳门, 未门, 申门, 戌门, 亥门:
    pair foo.i;
  endfor;
  foo.西北角 := (ulcorner foo); 
  foo.东北角 := (urcorner foo);
  foo.东南角 := (lrcorner foo);
  foo.西南角 := (llcorner foo);
  foo.子门 := .5[foo.西北角, foo.东北角];
  foo.午门 := .5[foo.西南角, foo.东南角];
  foo.卯门 := .5[foo.东南角, foo.东北角];
  foo.酉门 := .5[foo.西南角, foo.西北角];
  foo.丑门 := .5[foo.子门, foo.东北角];
  foo.寅门 := .5[foo.卯门, foo.东北角];
  foo.辰门 := .5[foo.卯门, foo.东南角];
  foo.巳门 := .5[foo.午门, foo.东南角];
  foo.未门 := .5[foo.午门, foo.西南角];
  foo.申门 := .5[foo.酉门, foo.西南角];
  foo.戌门 := .5[foo.酉门, foo.西北角];
  foo.亥门 := .5[foo.子门, foo.西北角];
  if 或显或隐:
    forsuffixes i = 东北角, 东南角, 西南角, 西北角:
      draw foo.i withpen pensquare scaled 4pt withcolor darkblue;
    endfor;
    forsuffixes i = 子门, 卯门, 午门, 酉门:
      draw foo.i withpen pencircle scaled 4pt withcolor darkred;
    endfor;
    forsuffixes i = 丑门, 寅门, 辰门, 巳门, 未门, 申门, 戌门, 亥门:
      draw foo.i
        withpen pencircle scaled 3pt withcolor darkgreen;
    endfor;
  fi;
enddef;
\stopMPinclusions

\startMPinclusions[+]
def 偏 expr a = shifted a enddef;
tertiarydef a 位于 b = a 偏 (center b - center a) enddef;
def 令 suffix a = 令之体(a) enddef;
def 令之体 (suffix a) text b = a := a b enddef;
\stopMPinclusions

\startMPinclusions[+]
pair 竖亥;
tertiarydef a 经转 b =
  hide(竖亥 := (xpart (if path a: point (length a) of a else: a fi), ypart (b)))
  a -- (point -(获 "配置.路径.圆角") on (a -- 竖亥))
    .. controls 竖亥 .. (point (获 "配置.路径.圆角") on (竖亥 -- b)) -- b
enddef;
tertiarydef a 纬转 b =
  hide(竖亥 := (xpart (b), ypart (if path a: point (length a) of a else: a fi)))
  a -- (point -(获 "配置.路径.圆角") on (a -- 竖亥))
    .. controls 竖亥 .. (point (获 "配置.路径.圆角") on (竖亥 -- b)) -- b
enddef;
\stopMPinclusions

\startMPinclusions[+]
tertiarydef a => b =
  begingroup
    save outgoing, incoming, va, vb, do_nothing;
    pair outgoing, incoming, va[], vb[];
    boolean do_nothing; do_nothing := false;
    va[1] := if pair a: a else: llcorner a fi;
    va[2] := if pair a: a else: urcorner a fi;
    vb[1] := if pair b: b else: llcorner b fi;
    vb[2] := if pair b: b else: urcorner b fi;
    if xpart va[2] < xpart vb[1]: % a 在 b 的左侧
      outgoing := if pair a: a else: 0.5[lrcorner a, urcorner a] fi;
      incoming  := (xpart vb[1], ypart outgoing);
    elseif xpart va[1] > xpart vb[2]: % a 在 b 的右侧
      outgoing := if pair a: a else: 0.5[llcorner a, ulcorner a] fi;
      incoming  := (xpart vb[2], ypart outgoing);
    elseif ypart va[1] > ypart vb[2]: % a 在 b 的上方
      outgoing := if pair a: a else: 0.5[llcorner a, lrcorner a] fi;
      incoming  := (xpart outgoing, ypart vb[2]);
    elseif ypart va[2] < ypart vb[1]: % a 在 b 的下方
      outgoing := if pair a: a else: 0.5[ulcorner a, urcorner a] fi;
      incoming  := (xpart outgoing, ypart vb[1]);
    else:
      do_nothing := true;
    fi;
    if do_nothing: nullpicture else: outgoing -- incoming fi
  endgroup
enddef;

drawpathoptions(withpen pencircle scaled (获 "配置.路径.线宽") withcolor (获 "配置.路径.颜色"));
def 流向 text a =
  drawarrowpath if (获 "配置.玩笑") > 0: (a) randomized (获 "配置.玩笑") else: a fi;
enddef;
def 串联 text a =
  drawpath if (获 "配置.玩笑") > 0: (a) randomized (获 "配置.玩笑") else: a fi;  
enddef;
\stopMPinclusions

\startMPinclusions[+]
def 标注 (expr tag, anchor, c) text p =
  begingroup
    save formatted_tag, pos, offset, t;
    pair pos; numeric offset; string t, formatted_tag;
    formatted_tag := "\tfx" & tag;
    pos := point c along (p);
    offset := .25(获 "配置.字号");
    if anchor = "北":
      pos := pos shifted (0, offset);
      t := "thetextext" & ".top";
    elseif anchor = "东":
      pos := pos shifted (offset, 0);
      t := "thetextext" & ".rt";
    elseif anchor = "南":
      pos := pos shifted (0, -offset);
      t := "thetextext" & ".bot";
    elseif anchor = "西":
      pos := pos shifted (-offset, 0);
      t := "thetextext" & ".lft";
    fi;
    draw scantokens(t)(formatted_tag, pos);
  endgroup;
enddef;
\stopMPinclusions

\startMPinclusions[+]
def 开启虚线模式 =
  drawpathoptions(dashed (evenly scaled .625(获 "配置.路径.线宽"))
                  withpen pencircle scaled (获 "配置.路径.线宽")
                  withcolor (获 "配置.路径.颜色"));
enddef;
def 关闭虚线模式 =
  drawpathoptions(withpen pencircle scaled (获 "配置.路径.线宽")
                  withcolor (获 "配置.路径.颜色"));
enddef;
% 默认关闭虚线模式
关闭虚线模式;
\stopMPinclusions

\startMPinclusions[+]
def 北 = up enddef; def 南 = down enddef;
def 西 = left enddef; def 东 = right enddef;
def 北行 expr a = (北 * (a)) enddef;
def 南行 expr a = (南 * (a)) enddef;
def 西行 expr a = (西 * (a)) enddef;
def 东行 expr a = (东 * (a)) enddef;

def 从 = enddef;
tertiarydef a 向 b =
  if pair a:
    a -- (a shifted b)
  elseif path a:
    a -- (point (length a) of a) shifted (b)
  fi
enddef;
\stopMPinclusions