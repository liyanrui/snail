\startluacode
local 蜗界 = {}
蜗界.字号 = "BodyFontSize" -- ConTeXt 正文字体所用字号
蜗界 = {
    字号 = 蜗界.字号,
    文字 = {颜色 = "black"},
    框形 = "fullsquare",
    框 = {余地 = 蜗界.字号, 
          线宽 = ".175" .. 蜗界.字号, 
          颜色 = "darkgray",
          各向同性 = "false",
          郊 = "2" .. 蜗界.字号},
    背景 = {颜色 = "lightgray"},
    路径 = {线宽 = ".2" .. 蜗界.字号,
            颜色 = "darkgray",
            有向 = "true",
            圆角 = ".25" .. 蜗界.字号},
    玩笑 = "0pt"
}
table.save("snail.conf", 蜗界)
\stopluacode

\startMPinclusions
lua("蜗界 = table.load('snail.conf')");
def 获 expr a = lua("mp.print(" & a & ")") enddef;
def 设 expr a = lua(a) enddef;
def 恢复默认蜗界 =
  lua("蜗界 = table.load('snail.conf')")
enddef;
\stopMPinclusions

\startMPinclusions[+]
def 中央四角十二门 (suffix foo)= 
  begingroup
  save t, 方向, 线宽; path t; pair 方向; numeric 线宽;
  forsuffixes i = 中央, 东北角, 东南角, 西南角, 西北角,
                  子门, 卯门, 午门, 酉门,
                  丑门, 寅门, 辰门, 巳门, 未门, 申门, 戌门, 亥门:
    pair foo.i;
  endfor;
  foo.中央 := (center foo);
  foo.西北角 := (ulcorner foo); 
  foo.东北角 := (urcorner foo);
  foo.东南角 := (lrcorner foo);
  foo.西南角 := (llcorner foo);
  foo.子门 := .5[foo.西北角, foo.东北角];
  foo.午门 := .5[foo.西南角, foo.东南角];
  foo.卯门 := .5[foo.东南角, foo.东北角];
  foo.酉门 := .5[foo.西南角, foo.西北角];
  foo.丑门 := .5[foo.子门, foo.东北角];
  foo.寅门 := .5[foo.卯门, foo.东北角];
  foo.辰门 := .5[foo.卯门, foo.东南角];
  foo.巳门 := .5[foo.午门, foo.东南角];
  foo.未门 := .5[foo.午门, foo.西南角];
  foo.申门 := .5[foo.酉门, foo.西南角];
  foo.戌门 := .5[foo.酉门, foo.西北角];
  foo.亥门 := .5[foo.子门, foo.西北角];
  %%%%%%%% 对于宫而言, 若框为异形,将角点和门投射到框的边界上
  if known foo.框:
    线宽 := 获 "蜗界.框.线宽";
    forsuffixes i = 东北角, 东南角, 西南角, 西北角,
                    子门, 卯门, 午门, 酉门,
                    丑门, 寅门, 辰门, 巳门, 未门, 申门, 戌门, 亥门:
      foo.i := 5(foo.i shifted -foo.中央) shifted foo.中央;
      方向 := unitvector(foo.i - foo.中央);
      t := (foo.中央 -- foo.i) cutafter foo.框;
      foo.i := .5线宽 * 方向 + point (length t) of t ;
    endfor;
  fi;
enddef;
def 郊 (suffix foo) = 
  forsuffixes i = 乾位, 坤位, 艮位, 兑位, 巽位, 震位, 坎位, 离位:
    pair foo.i;
  endfor;
  begingroup
    save outskirts; path outskirts;
  outskirts := foo enlarged (获 "蜗界.框.郊");
  foo.艮位 := (ulcorner outskirts); 
  foo.震位 := (urcorner outskirts);
  foo.兑位 := (lrcorner outskirts);
  foo.巽位 := (llcorner outskirts);
  foo.坤位 := .5[foo.艮位, foo.震位];
  foo.乾位 := .5[foo.巽位, foo.兑位];
  foo.离位 := .5[foo.兑位, foo.震位];
  foo.坎位 := .5[foo.巽位, foo.艮位];
enddef;

def 关防要塞 text all =
  forsuffixes foo = all:
    forsuffixes i = 乾位, 坤位, 艮位, 兑位, 巽位, 震位, 坎位, 离位:
      draw foo.i withpen pensquare scaled 4pt withcolor darkgreen;
    endfor;
    forsuffixes i = 子门, 卯门, 午门, 酉门:
      draw foo.i withpen pencircle scaled 4pt withcolor darkred;
    endfor;
    forsuffixes i = 丑门, 寅门, 辰门, 巳门, 未门, 申门, 戌门, 亥门:
      draw foo.i withpen pencircle scaled 2pt withcolor magenta;
    endfor;
    forsuffixes i = 东北角, 东南角, 西南角, 西北角:
      draw foo.i withpen pensquare scaled 4pt withcolor darkblue;
    endfor;
  endfor;
enddef;
\stopMPinclusions

\startMPinclusions[+]
def 原点 = (0, 0) enddef;
def 偏 = shifted enddef;
tertiarydef a 位于 b =
  (if pair b: b else: center b fi - if pair a: a else: center a fi)
enddef;
def 定位 (suffix name) (expr 多少呢) =
  name := name 偏 (多少呢);
  中央四角十二门(name); 郊(name);
enddef;
\stopMPinclusions

\startMPinclusions[+]
def 宫 (suffix name) (expr a) text 变换语句 =
  picture name;
  begingroup
  save w, h, s, delta;
  path name.框; picture name.文; numeric w, h, s; pair delta;
  name.文 := textext(a);
  w := bbwidth name.文; h := bbheight name.文;
  if (获 "蜗界.框.各向同性"):
    if w > h: s = w; else: s = h; fi;
    name.框 := fullsquare xysized (s, s) enlarged ((获 "蜗界.框.余地") * (1, 1));
  else:
    name.框 := fullsquare xysized (w, h) enlarged ((获 "蜗界.框.余地") * (1, 1));
  fi;
  name.框 := (获 "蜗界.框形") xysized (bbwidth name.框, bbheight name.框);
  if (获 "蜗界.玩笑") > 0: name.框 := name.框 randomized (获 "蜗界.玩笑"); fi;
  name := image (fill name.框 withcolor (获 "蜗界.背景.颜色");
                 draw name.框 withpen pencircle scaled (获 "蜗界.框.线宽")
                         withcolor (获 "蜗界.框.颜色");
                 draw name.文 withcolor (获 "蜗界.文字.颜色"););
  delta := (0, 0) 变换语句;
  name := name 偏 delta;
  name.框 := name.框 偏 delta;
  中央四角十二门(name); 郊(name);
  endgroup
enddef;
\stopMPinclusions

\defineframed[snailframed][offset=1em,frame=off]
\startMPinclusions[+]
def 廷 (suffix name) (expr a) text 变换语句 =
  picture name;
  begingroup
    picture name.文; name.文 := textext("\snailframed{" & a & "}");
    name := image (draw name.文 withcolor (获 "蜗界.文字.颜色"););
    name := name 偏 ((0, 0) 变换语句); 
    中央四角十二门(name); 郊(name);
  endgroup
enddef;
\stopMPinclusions

\startMPinclusions[+]
def 宫廷 (suffix a) (expr b) = picture a; a := b; enddef;
def 城 (suffix a) (text b) =
  宫廷(a, nullpicture);
  for i = b: addto a also image(draw i); endfor;
  中央四角十二门(a, 隐); 郊(a, 隐);
enddef;
\stopMPinclusions

\startMPinclusions[+]
def 景 (suffix name) (expr a, w, h) text 变换语句 =
  picture name;
  begingroup;
  save p, s; picture p; numeric s;
  p := externalfigure a;
  s := (bbwidth p) / (bbheight p);
  if numeric w and numeric h:
    name := image(draw externalfigure a xysized (w, h););
  else:
    if numeric w:
      name := image(draw externalfigure a xysized (w, w / s););
    fi;
    if numeric h:
       name := image(draw externalfigure a xysized (h * s, h););
    fi;
  fi;
  name := name 偏 -(center name); % 图像默认是左下角点位于原点
  name := name 偏 ((0, 0) 变换语句); 
  中央四角十二门(name); 郊(name);
  endgroup
enddef;

def 呜呼 text all =
  for i = all: draw i; endfor;
enddef;
\stopMPinclusions

\startMPinclusions[+]
pair 竖亥;
tertiarydef a 经转 b =
  hide(竖亥 := (xpart (if path a: point (length a) of a else: a fi), ypart (b)))
  a -- 竖亥 -- b
enddef;
tertiarydef a 纬转 b =
  hide(竖亥 := (xpart (b), ypart (if path a: point (length a) of a else: a fi)))
  a --  竖亥 -- b
enddef;
\stopMPinclusions

\startMPinclusions[+]
vardef 经距 (expr a, b) = (abs(xpart (a) - xpart (b))) enddef;
vardef 纬距 (expr a, b) = (abs(ypart (a) - ypart (b))) enddef;
\stopMPinclusions

\startMPinclusions[+]
drawpathoptions(withpen pencircle scaled (获 "蜗界.路径.线宽") withcolor (获 "蜗界.路径.颜色"));
def 流向 text a =
  drawarrowpath if (获 "蜗界.玩笑") > 0: (a) randomized (获 "蜗界.玩笑") else: a fi;
enddef;
def 串联 text a =
  drawpath if (获 "蜗界.玩笑") > 0: (a) randomized (获 "蜗界.玩笑") else: a fi;  
enddef;
\stopMPinclusions

\startMPinclusions[+]
def 开启虚线模式 =
  drawpathoptions(dashed (evenly scaled .625(获 "蜗界.路径.线宽"))
                  withpen pencircle scaled (获 "蜗界.路径.线宽")
                  withcolor (获 "蜗界.路径.颜色"));
enddef;
def 关闭虚线模式 =
  drawpathoptions(withpen pencircle scaled (获 "蜗界.路径.线宽")
                  withcolor (获 "蜗界.路径.颜色"));
enddef;
% 默认关闭虚线模式
关闭虚线模式;
\stopMPinclusions

\startMPinclusions[+]
def 北 expr a = a * up enddef; 
def 南 expr a = a * down enddef;
def 西 expr a = a * left enddef; 
def 东 expr a = a * right enddef;

tertiarydef a 纬至 b = 
  if pair a:
    a -- (a shifted (0, ypart b - ypart a))
  elseif path a:
    hide(竖亥 := (point (length a) of a))
    a -- (竖亥 shifted (0, ypart b - ypart 竖亥))
  fi
enddef;
tertiarydef a 经至 b = 
  if pair a:
    a -- (a shifted (xpart b - xpart a, 0))
  elseif path a:
    hide(竖亥 := (point (length a) of a))
    a -- (竖亥 shifted (xpart b - xpart 竖亥, 0))
  fi
enddef;
\stopMPinclusions

\startMPinclusions[+]
vardef 共线 (expr p, a) =
  save 结果, v; boolean 结果; pair v[];
  v1 := unitvector((point 0 of p) - (a));
  v2 := unitvector((point (length p) of p) - (a));
  if abs((v1 dotprod v2) - 1) < 1e-6:
     结果 := true;
  else:
    结果 := false;
  fi;
  结果
enddef;

vardef 简化 (expr p) =
  save n, it, new_p, c;
  numeric n; path it, new_p; pair c;
  n := length p;
  it := point 0 of p;
  new_p := it;
  for i = 1 upto n:
    c := point i of p;
    if (length it) = 0: 
      it := it -- c;
    else:
      if 共线(it, c):
        it := it -- c;
      else:
        new_p := new_p -- (point (length it) of it);
        it := (point (length it) of it) -- c;
      fi;
    fi;
  endfor;
  if (length it) > 0:
    new_p := new_p -- point (length it) of it;
  fi;
  new_p
enddef;

vardef 柔和 (expr p) =
  save n, nodes, new_p, 圆角; 
  numeric n, 圆角; pair nodes[]; path new_p;
  n := length p;
  for i = 0 upto n:
    nodes[i] := point i of p;
  endfor;
  new_p := nodes[0];
  圆角 := (获 "蜗界.路径.圆角");
  for i = 1 upto n - 1:
    new_p := new_p -- point -圆角 on (nodes[i - 1] -- nodes[i]);
    new_p := new_p .. point 圆角 on (nodes[i] -- nodes[i + 1]);
  endfor;
  new_p := new_p -- nodes[n];
  new_p
enddef;
\stopMPinclusions

\startMPinclusions[+]
vardef 最长一段的中点 (expr p) =
  save n, l, l_max, s_max, t;
  numeric n, l, l_max; path t, s_max;
  n := length p;
  l_max := 0;
  for i = 1 upto n:
    t := (point i - 1 of p) -- (point i of p);
    l := arclength t;
    if l_max < l:
      l_max := l; s_max := t;
    fi;
  endfor;
  point .5 along (s_max)
enddef;

def 标注之体 (expr tag, anchor, c) text p =
  begingroup
    save formatted_tag, pos, offset, t;
    pair pos; numeric offset; string t, formatted_tag;
    formatted_tag := "\tfx" & tag;
    if c < 0:
      pos := 最长一段的中点 p;
    else:
      pos := point c along p;
    fi;
    offset := .25(获 "蜗界.字号");
    if anchor = "北":
      pos := pos shifted (0, offset);
      t := "thetextext.top";
    elseif anchor = "东":
      pos := pos shifted (offset, 0);
      t := "thetextext.rt";
    elseif anchor = "南":
      pos := pos shifted (0, -offset);
      t := "thetextext.bot";
    elseif anchor = "西":
      pos := pos shifted (-offset, 0);
      t := "thetextext.lft";
    fi;
    draw scantokens(t)(formatted_tag, pos);
  endgroup
enddef;
vardef 标注(expr tag, anchor) text p =
  标注之体 (tag, anchor, -1) (p);
  (0, 0) -- (0, 0)
enddef;
vardef 定位标注(expr tag, anchor, c) text p =
  标注之体 (tag, anchor, c) (p);
  (0, 0) -- (0, 0)
enddef;
\stopMPinclusions

\startMPinclusions[+]
def 路 (suffix name) (expr p) text label =
  begingroup
  save hack; path hack;
  path name; name := 柔和(简化(p));
  if (获 "蜗界.路径.有向"):
    流向 name;
  else:
    串联 name;
  fi;
  hack := label name;
  if path hack: fi;
  endgroup
enddef;
\stopMPinclusions

\startMPinclusions[+]
vardef 单段曲线 (expr p) =
  numeric n; n := length p;
  if n = 1:
    new_p := (point 0 of p .. point 1 of p);
  elseif n = 2:
    new_p := (point 0 of p .. controls (point 1 of p) .. point 2 of p);
  elseif n = 3:
    new_p := (point 0 of p .. 
        controls (point 1 of p) and (point 2 of p) 
        .. point 3 of p);
  else:
    new_p := p;
  fi;
  new_p
enddef;
vardef 小曲线 (expr p) =
  save m, n, k, new_p, segments, nodes;
  numeric m, n, k; path new_p, segments[]; pair nodes[];
  n := length p;
  if n < 4:
    new_p := 单段曲线(p);
  else:
    new_p := 简化(p);
    m := length new_p;
    for i = 0 upto m:
      nodes[i] := point i of p;
    endfor;
    %%%% 分段
    % 首
    new_p := nodes[0] 
          .. controls nodes[1]
          .. .5[nodes[1], nodes[2]]
          .. controls nodes[2]
          .. .5[nodes[2], nodes[3]];
    % 体
    m := n - 4;
    k := 1;
    for i = 2 step 2 until m:
      new_p := new_p
          .. controls nodes[i + 1]
          .. .5[nodes[i + 1], nodes[i + 2]]
          .. controls nodes[i + 2]
          .. .5[nodes[i + 2], nodes[i + 3]];
      k := i + 2;
    endfor;
    % 尾:只可能有 3 个或 4 个控制点
    show n, k;
    m := n - k + 1;
    if m = 3:
      new_p := new_p .. controls nodes[k + 1] .. nodes[n];
    elseif m = 4:
      new_p := new_p 
        .. controls nodes[k + 1]
        .. .5[nodes[k + 1], nodes[k + 2]]
        .. controls nodes[k + 2]
        .. nodes[n];
    else:
      show "小曲线出错!";
    fi;
  fi;
  new_p
enddef;
vardef 曲线 (expr p) =
  save m, n, k, new_p, nodes, segments;
  numeric m, n, k; path new_p, segments[]; pair nodes[];
  n := length p;
  if n < 4:
    new_p := 单段曲线(p);
  else:
    new_p := 简化(p);
    m := length new_p;
    for i = 0 upto m:
      nodes[i] := point i of p;
    endfor;
    % 分段
    k := 0;
    n := m - 2;
    for i = 0 step 2 until n:
      if i = 0:
        segments[k] := nodes[i] 
          .. controls nodes[i + 1] and nodes[i + 2] 
          .. .5[nodes[i + 2], nodes[i + 3]];
      elseif i + 2 = m:
        segments[k] := .5[nodes[i], nodes[i + 1]] 
          .. controls nodes[i + 1]
          .. nodes[i + 2];
      elseif i + 3 = m:
        segments[k] := .5[nodes[i], nodes[i + 1]] 
          .. controls nodes[i + 1] and nodes[i + 2] 
          .. nodes[i + 3];
      else:
        segments[k] := .5[nodes[i], nodes[i + 1]] 
          .. controls nodes[i + 1] and nodes[i + 2] 
          .. .5[nodes[i + 2], nodes[i + 3]];
      fi;
      k := k + 1;
    endfor;
    % 合并
    new_p := segments[0];
    for i = 1 upto k - 1:
      new_p := new_p .. segments[i];
    endfor;
  fi;
  new_p
enddef;
def 弯路之体 (suffix name) (expr p, 方法) text label =
  begingroup
  save hack; path hack;
  path name; name := scantokens(方法)(简化(p));
  if (获 "蜗界.路径.有向"):
    流向 name;
  else:
    串联 name;
  fi;
  hack := label name;
  if path hack: fi;
  endgroup
enddef;
def 小弯路 (suffix name) (expr p) text label =
  弯路之体(name, p, "小曲线") label;
enddef;
def 弯路 (suffix name) (expr p) text label =
  弯路之体(name, p, "曲线") label;
enddef;
\stopMPinclusions

\startMPinclusions[+]
def 度 (suffix a) (expr b) =
  numeric a;
  a := b;
enddef;
\stopMPinclusions
