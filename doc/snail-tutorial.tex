\environment doc-style
\starttext
\title{Snail: A Diagram Module for \ConTeXt\ \crlf via MetaPost}

\subject{Introduction}

Maybe everyone knows MetaPost is great at drawing accurate science diagrams. But that's not the whole story! Think of it this way, it's actually a programming language for drawing vector graphics with simple syntax. You can write code to create the same great things as in those industry software (e.g., Adobe Illustrator, CorelDraw or Inkscape).

Don't believe the above claim I made. Let's be real. I've never worked as a graphic designer. Visual tools are much faster and easier for design work than coding with MetaPost. Exactly like how most pepole prefer movies over books these days for storytelling.

However, in this technological era, I wrote a compact MetaPost module for drawing flowchart and diagrammatic illustrations. Is this somehow a lamentable act? Certainly not. I have never intended to resist technological progress. Our era pursues its trajectory; I pursue mine. The primary driver for writing this module was the absence of satisfactory flowchart software on my Linux desktop.

This module is named \boxquote{snail}. True to its name, it's slow at drawing diagrams.

Its development has been even slower. Around 2018, while properly learning MetaPost for the first time, I wrote the version 1 as practice work. It did not work well. In fact, after finishing it, I showed it off to a few friends then never used it again. In 2023, I relearned MetaPost and created the version 2. This one actually worked---I'd say it worked well. But I discovered MetaPost supports Chinese variable and macro names, so I built this version as a Chinese diagramming language. Both the variables and macros got quirky Chinese names. That's why I've never showed it to anyone since finishing it.

\useURL[ctxnotes][https://github.com/liyanrui/ConTeXt-notes]

Now we've reached 2025. These past seven years? Honestly I've achieved nothing remarkable. Feeling low, I revisited MetaPost and wrote some documentation. This finally fulfilled my wish. Two years ago, after finishing \ConTeXt\ notes\footnote{Please see \boxquote{\from[ctxnotes]}.}, I'd wanted to write similar guides for MetaPost language. Along the way, third version of snail took shape. Maybe this is all I'm capable of, simple tools for simple needs.

\section{Your First Snail Diagram}

This is the smallest possible snail drawing environment. All that's left is to write some MetaPost code inside the \type{MPpage} environment.

\startTEX
\usemodule[snail]
\startMPpage
% put metapost code here!
\stopMPpage
\stopTEX

To compile a \TEX\ source file foo.tex into foo.pdf, use

\starttyping
/BTEX\dollar/ETEX context foo.tex
\stoptyping

or

\starttyping
/BTEX\dollar/ETEX context foo
\stoptyping

Then you can get the foo.pdf file in the same directory. The above process can be expressed as the following snail code.

\startsample
\usemodule[snail]
\startMPpage
snailfam_t a[];
slug(a1, "foo.tex");
snail(a2, "context") at (2.5cm, 0);
slug(a3, "foo.pdf") at (5cm, 0);
showsnails a1, a2, a3;

showflow a1 xto a2;
showflow a2 xto a3;
\stopMPpage
\stopsample
\sample[option=TEX][first-diagram]{Your first diagram}{\externalfigure[figures/01.pdf]}

\section{Node Declaration}

All snail objects, i.e. nodes, must be declared before defining them. There are two declaration methods. The first one declares a group of nodes with the macro \type{snail_t}. The other one declares a node sequence (or array) with the macro \type{snailfam_t}.

For example, to declare the nodes \type{foo} and \type{bar}, use

\starttyping
snail_t foo, bar;
\stoptyping

To create a sequence of nodes, use MetaPost's array syntax. For example,

\starttyping
snailfam_t A[];
\stoptyping

The above declares a node array that the elements are accessible with indices, i.e. \type{A[1]}, \type{A[2]}, \type{A[3]}, \cdots\  or  \type{A1}, \type{A2}, \type{A3}, \cdots

Critically, note that each \type{snailfam_t} declaration can't declare multiple arrays at once. The following code is wrong!

\starttyping
snailfam_t A[], B[];
\stoptyping

Instead, you must declare each array separately like this:

\starttyping
snailfam_t A[];
snailfam_t B[];
\stoptyping

Use MetaPost's \type{forsuffixes} syntax to declare more arrays at once. For example, 

\starttyping
forsuffixes it = A, B, C, D, E, F:
  snailfam_t it[];
endfor;
\stoptyping

or

\starttyping
forsuffixes it = A, B, C, D, E, F: snailfam_t it[]; endfor;
\stoptyping

It declares \type{A[]}, \type{B[]}, \cdots\ , \type{F[]}.

\stoptext
